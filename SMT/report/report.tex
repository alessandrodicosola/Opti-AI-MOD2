\documentclass{report}

\usepackage{hyperref}
\usepackage{geometry}
\usepackage{listings}
\usepackage{graphicx,wrapfig}
\usepackage{subcaption}

\usepackage{lmodern}
\usepackage[T1]{fontenc}
\usepackage{textcomp}

\date{}
\geometry{top=1cm}
\begin{document}
\begin{titlepage}
    \vspace*{1cm}

    \begin{center}
        \Huge{Alma Mater Studiorum 
        
        University of Bologna }
    \end{center}

    \vspace*{5 cm}
    \begin{center}
        \LARGE{Combinatorial Decision Making and Optimization}

        \Large Report SMT Model

        \large {Dicosola Alessandro [Matr. 935563]}
    \end{center}
\end{titlepage}

\section*{Model}
The model use the same encoding as CP: X and Y as decision variables.
In order to take into account the rotation also $Wi$ and $Hi$ are used for constraning the width and height of each rectangle:
\begin{lstlisting}[basicstyle=\footnotesize]
    X = IntVector("x",N)
    Y = IntVector("y",N)
    Wi = IntVector("w",N)
    Hi = IntVector("h",N)
\end{lstlisting} 
%
The constraints used are:
\begin{itemize}
    \item Initialize $w_i$,$h_i$ for each rectangle
    
    \begin{math}
        \bigwedge_{i = 1}^{N}  w_i = W_i \wedge  h_i = H_i \vee h_i = W_i \wedge w_i = H_i 
    \end{math}
    \item X and Y should be bounded within the "available" area given their size
    
    \begin{math}
        \bigwedge_{i = 1}^{N}  x_i >= 0 \wedge  x_i <= W - w_i \wedge y_i >= 0 \wedge y_i <= H - h_i 
    \end{math}
    
    \item No overlapping constraint:

    Two rectangles $i$ ad $j$ overlap if:
    %
    \begin{math}
        x_i+w_i>x_j \wedge x_i<x_j+w_j \wedge y_i+h_i>y_j \wedge y_i<y_j+h_j     
    \end{math}
    %
    thus negating it we have the no-overlap constraint:

    \begin{math}
        \bigwedge_{i = 1}^{N} x_i+w_i \leq x_j \vee x_j+w_j \leq x_i \vee y_i+h_i \leq y_j \vee y_j+h_j \leq y_i
    \end{math}
    \item In case of rectangles with same size I' ve created the \textbf{lex\_lesseq} constraint decomposing it.
    \item The implicit constraint: given a vertical (resp. horizontal) line the sum of each vertical (resp. horizontal) edge is H (resp. W).
\end{itemize}
%
Both \textbf{rotation} and \textbf{rectangles with same size} are handled althought for the latter case when there aren't rectangles with same size \textit{unsat} is returned thus this constraint is activated only manually by command line.
\end{document}



